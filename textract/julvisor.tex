\begin{flushleft}
{\Huge Julvisor\\}
{\Large
\vspace{1cm}
Julen infaller 24 december varje år och firas till minne av Kristi
födelse. Lucia infaller mitt i tentaveckan i läsperiod två och firas
till minne av Lucias död.}
\end{flushleft}
\vspace{2cm}
\begin{center}
\includegraphics[width=6cm]{bilder/112.eps}

\end{center}

\newpage

\begin{song}{Hej tomtegubbar}{hejtomtegubbar}
\begin{vers}
Hej tomtegubbar slå i glasen\\
och låt oss lustiga vara.\\
Hej tomtegubbar slå i glasen\\
och låt oss lustiga vara.\\
En liten tid vi leva här\\
med mycken möda och stort besvär.\\
Hej tomtegubbar slå i glasen\\
och låt oss lustiga vara.\\
\end{vers}
\end{song}

\begin{song}{Nu har vi ljus}{nuharviljus}
\begin{vers}
Nu har vi ljus här i vårt hus,\\
julen är kommen, hopp fa-ra-la-la!\\
Barnen i ring dansa omkring,\\
dansa omkring.\\
Granen står så grön och grann i stugan,\\
granen står så grön och grann i stugan.\\
Tra-la-la-la-la, tra-la-la-la-la,\\
ra-la-la-la-la, la-la!\\
\end{vers}
\end{song}
\newpage

\begin{song}{Nu är det jul igen}{nuardetjuligen}
\begin{vers}
//: Nu är det jul igen, och nu är det jul igen,\\
och julen varar än till påska. ://\\
//: Men det var inte sant, och det var inte sant,\\
för däremellan kommer fasta. ://\\
\end{vers}
\end{song}


\begin{song}{Rudolf med röda mulen}{rudolfmedrodamulen}
\begin{vers}
Rudolf med röda mulen\\
hette en helt vanlig ren\\
som blivit kall om mulen,\\
därav kom dess röda sken.\\
Rudolf fick alltid höra\\
"Se, han har sitt dimljus på",\\
att han blev led på detta,\\
det är sånt man kan förstå.\\
\end{vers}
\begin{vers}
Men en mörk julaftonskväll\\
Tomtefar han sa:\\
"Vill du inte Rudolf säg,\\
med din mule lysa mig."\\
Allt se'n den da'n den renen\\
tomtens egen släde drar.\\
Rudolf med röda mulen\\
lyser väg för tomtefar.\\
\end{vers}
\end{song}

\newpage

\begin{song}{Sankta Lucia}{sanktalucia}
\begin{vers}
Natten går tunga fjät runt gård och stuva.\\
Kring jord som sol förgät, skuggorna ruva.\\
Då i vårt mörka hus, stiger med tända ljus,\\
Sankta Lucia, Sankta Lucia.\\
\end{vers}
\begin{vers}
Natten var stor och stum. Nu hörs det svingar,\\
i alla tysta rum, sus som av vingar.\\
Se på vår tröskel står vitkläd, med ljus i hår,\\
Sankta Lucia, Sankta Lucia.\\
\end{vers}
\end{song}

\begin{song}{Glöggvisa}{gloggvisa}
\mel{Och jungfrun hon går i dansen med röda gullband}
\begin{vers}
Ack, glöggen den står på bordet och ångar sig varm\\
vi häller den ner i halsen och värmer vår tarm.\\
En här, en där, utan minsta besvär\\
den drycken är vår strupe så innerligt kär\\
\end{vers}
\end{song}

\begin{song}{Nu tändas tusen juleljus}{nutandas}
\begin{vers}
Nu tändas tusen juleljus på jordens mörka rund.\\
Och tusen, tusen strålar ock på himlens djupblå grund.\\
För över stad och land i kväll går julens glada bud.\\
Att född är herren Jesus Krist, vår frälsare och Gud.
\end{vers}
\end{song}

\begin{song}{Staffansvisan}{staffansvisan}
\begin{vers}
Staffan var en stalledräng,\\
vi tackom nu så gärna.\\
Han vattnade sina fålar fem,\\
allt för den ljusa stjärna.\\
Ingen dager synes än,\\
stjärnorna på himmelen de blänka.\\
\end{vers}
\begin{vers}
Två de voro röda, vi tackom nu så gärna.\\
de tjänte väl sin föda, allt för den ljusa stjärna.\\
Ingen dager...\\
\end{vers}
\begin{vers}
Två de voro vita, vi tackom nu så gärna.\\
de va varandra lika, allt för den ljusa stjärna.\\
Ingen dager...\\
\end{vers}
\begin{vers}
Den femte den var apelgrå, vi tackom nu så gärna.\\
den rider Staffan själv uppå, allt för den ljusa stjärna.\\
Ingen dager...\\
\end{vers}
\begin{vers}
Innan hanen galigt har, vi tackom nu så gärna.\\
Staffan uti stallet var, allt för den ljusa stjärna.\\
Ingen dager...\\
\end{vers}
\end{song}
\newpage

\begin{song}{Stilla natt}{sillanatt}
\begin{vers}
Stilla natt, heliga natt,\\
allt är frid, stjärnan blid\\
skiner på barnet i stallets strå,\\
och de vakande fromma två.\\
Kristus till jorden är kommen,\\
oss är en frälsare född.\\
\end{vers}
\begin{vers}
Stora stund, heliga stund,\\
änglars här går sin rund\\
kring de vaktande herdarnas hjord.\\
Rymden ljuder av glädjens ord:\\
Kristus till jorden är kommen,\\
eder är frälsaren född.\\
\end{vers}
\begin{vers}
Stilla natt, heliga natt,\\
mörkret flyr, dagen gryr.\\
Räddningstimmen för världen slår,\\
nu begynner vårt ljubelår.\\
Kristus till jorden är kommen,\\
oss är en frälsare född.\\
\end{vers}
\end{song}

\newpage

\begin{song}{Tomtarnas julnatt}{tomtarnasjulnatt}
\begin{vers}
Midnatt råder, tyst det är i husen, tyst i husen.\\
Alla sova, släckta äro ljusen, äro ljusen.\\
Tipp-tapp, tipp-tapp,\\
tippe-tippe-tipp-tapp.\\
Tipp, tipp, tapp.\\
\end{vers}
\begin{vers}
Se, då krypa tomtar upp ur vrårna, upp ur vrårna.\\
Lyssna, speja, trippa fram på tårna, fram på tårna.\\
Tipp-tapp...\\
\end{vers}
\begin{vers}
Snälla folket låtit maten rara, maten rara,\\
stå på bordet åt en tomteskara, tomteskara.\\
Tipp-tapp...\\
\end{vers}
\begin{vers}
Hur de mysa, hoppa upp bland faten, upp bland faten.\\
Tissla, tassla: "God är julematen, julematen.\\
Tipp-tapp...\\
\end{vers}
\begin{vers}
Se, då krypa tomtar upp ur vrårna, upp ur vrårna.\\
Lyssna, speja, trippa fram på tårna, fram på tårna.\\
Tipp-tapp...\\
\end{vers}
\begin{vers}
Snälla folket låtit maten rara, maten rara,\\
stå på bordet åt en tomteskara, tomteskara.\\
Tipp-tapp...\\
\end{vers}
\newp
\begin{vers}
Hur de mysa, hoppa upp bland faten, upp bland faten.\\
Tissla, tassla: "God är julematen, julematen".\\
Tipp-tapp...\\
\end{vers}
\begin{vers}
Gröt och skinka, lilla äppelbiten, äppelbiten.\\
Tänk så rart det smakar Nisse liten, Nisse liten.\\
Tipp-tapp...\\
\end{vers}
\begin{vers}
Nu till lekar, glada skrattet klingar, skrattet klingar.\\
Runt om granen skaran muntert svingar, muntert svingar.\\
Tipp-tapp...\\
\end{vers}
\begin{vers}
Natten lider. Snart de tomtar snälla, tomtar snälla,\\
kvickt och näpet allt i ordning ställa, i ordning ställa.\\
Tipp-tapp...\\
\end{vers}
\begin{vers}
Sedan åter in i tysta vrårna, tysta vrårna,\\
tomteskaran tassar nätt på tårna, nätt på tårna.\\
Tipp-tapp...\\
\end{vers}
\end{song}
